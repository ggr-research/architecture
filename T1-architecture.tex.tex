
This document describes basic principles of software architectures,
including architectures for IoT and cloud plattforms.
In essence, it is an overview of existing approaches in te FInest
and LoFIP project.


\section{FInest Architecture}

The goal of the FInest is to develop a Future Internet enabled ICT platform with a dedicated
purpose to enable the integration within transport and logistics domain.

The FInest platform should be customizable for specific business needs.

The FInest Platform offers a generic, configurable set of services
for the logistic domain.
FInest follows the TAM specification to describe architectures.
TAM is used on two different levels, the \emph{conceptual level} to describe
high-level architectures in low details and on \emph{design level}.

TAM is used at four different modeling levels, where at each level different
TAM diagrams are used.

\begin{itemize}
	\item The \textbf{initial conceptual architecture} is described by block diagrams.
	   It describes the overall architecture at the conceputal level.
	\item The \textbf{technical design level} extends the block diagrams with \emph{component diagrams}
	    to provide a more technical view on the architecture.
	\item The \textbf{detailed technological specification} use package and class diagrams
	      to give a more complete technical view.
	 \item Finally, \textbf{detailed examples} help to illustrate technical and conceptual details from 
	       the architecture. Examples are mainly described by sequence diagrams and class diagrams.
\end{itemize}
 

%%%%%%%%%%%%%%%%%%%%%%%%%%%%%%%%%%%%%%%%%%%%%%%%%%%
\subsection{Initial Conceputal Architecture}

The architecture should take into account the following design principles:
\begin{itemize}
	\item An extensible collection of components
	\item loosely coupled and interoperable components (components can be independently be selected and aggregated
	\item modular and decoupled design: every feature in a separate technical component
\end{itemize}


\subsubsection{High-level Architecture}

\begin{itemize}
	\item \textbf{Front End:} The front end layer provides a single point of user access via different devices.
	  There is an integrated user interface to offer access for various devices like web, mobile, desktop, etc.
	   The unser management and access control provide facilities for the registration of users and the
		  access control for users. Users (business end-users) should have access to all information that is relevant 
			for managing transport and logicstics processes. 
 \item \textbf{Core Modules:} The middle layer consists of the functional core moduls that offer
			the basic functionalities. All core modules are provided in form of  independent cloud-based services.
			These \emph{modules} should support the marketing, planning, execution and completion of logistics processes.
			In particular, there is a \emph{business collaboration module (BCM)} that keeps all information for the
			execution of a logistic  process where various different business partners are involved. 
			The \emph{E-contracting module (ECM)} provides computer support for service provider selection,
			contract negotiation and contract management. The \emph{Event processing module} enables real-time tracing
			and monitoring, as well as rule-based analysis of expected and unexpected events like triggering and re-planning
			when necessary.
			Finally, the transport planning module (TPM) is responsible for dynamic transport planning and re-planning.
				\begin{enumerate}
					\item The \emph{business collaboration module (BCM)} follow a data-centric modeling approach to
					   cover information about the overall transport and logistics chan.
						
						 This module supports the inter-organizational (business-2-business) collaboration between
						transport and logicstics network partners by tracking and tracing transports on the level of
						 business processes. 
					\item The \emph{E-Contracting Module (ECM)} is responsible for the contracting within the transport domain.
					        The service, which is provided by the ECM includes the integration of marketplaces, bidding,
									 negotiation, the execution of e-contracting selection and the management of contract selection
									  and execution. 
					          
										To achieve this, the architecture contains a data repository for contract storage,
										   an interface for contract demand manager and a contract manager for electronic contract
											access and partner selection.
					\item The \emph{Event processing module (EPM)} 		consists of a run-time engine and a set of rules.
					The input are events (from several sources) and the output are detected situations and status descriptions
					 of a running process. The EPM also takes care of event-driven monitoring and rule-based
					 analysis of expected and unexpected events (e.g., triggering re-planning).
					\item The \emph{Transport planning module (TPM)} consists of a planning engine to compose plans,
					                   to search for dedicated services and to apply simulation of consider the service behavior.
				\end{enumerate}
 \item \textbf{Back End:} The back end is the access part to various external
    legacy systems and standard business software, which are third party services and any
Internet of Things (IoT) devices that might provide information during the transport life-cycle.
\end{itemize}


\subsubsection{Requirements for the Architecture}


%%
\paragraph{Back-End Requirements}

The aim of the Back-end is to facilitate the integration of external systems (i.e., legacy systems,
standard business software and third-party software) to enable an automated
import and export of data into and from the platform.
The requirements analysis for the Back-End is based on existing
systems that the domain parters are using.


%%%%%%%%%%%%%%%%%%%%%%%%%%%%%%%%%%%%%%%%%%%%%%%%%%%%%%%%%%%%%%%%%%%%%%%%%%%%%

\section{Recognized Research Aspects}

\begin{itemize}
	\item Customization of Architectures or more precisely the
	   customization of configurable architectures.
		 (see the description on Customization tools in D3.2 sect. 2.1.2)
\end{itemize}