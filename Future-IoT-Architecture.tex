\documentclass{llncs} % , times

\usepackage{amssymb, amsmath, graphicx, ltxtable, longtable, tabularx, url, ragged2e, xspace, verbatim, fancybox,tikz}
\usepackage{scalefnt}
\usepackage{relsize}
\usepackage{paralist}
\usepackage{listings}

\newcommand{\ggr}[1]{\textcolor{magenta}{comment Gerd: \textit{#1}}}


\begin{document}
\title{Internet of Thinks for Transport and Logistics}
 

\maketitle

\begin{abstract}

\end{abstract}


\section{The Internet of Things}
The internet of things refers to a set of uniquely identifiable objects and their virtual representation in an internet-like structure. (RFID is often seen as a prerequisite for this technology.)


\paragraph{Background} The motivation for the Internet of Tings was that
peoples are the most numerous and important routers in the internet, in which
a Web is built of servers, routers and clients. Computers -- and, therefore, the internet --
are dependent on humans for information. However, people are not good
at capturing data from the Web and submit data to others in the Web.
To remedy this, RFID technology is one basic technology  to observe and understand the world.


\section{FInest Architecture}



\end{document}