\documentclass{llncs} % , times

\usepackage{amssymb, amsmath, graphicx, ltxtable, longtable, tabularx, url, ragged2e, xspace, verbatim, fancybox,tikz}
\usepackage{scalefnt}
\usepackage{relsize}
\usepackage{hyperref}
%\usepackage{paralist}
%\usepackage{listings}

\newcommand{\ggr}[1]{\textcolor{magenta}{comment Gerd: \textit{#1}}}

\pagenumbering{arabic}
\pagestyle{plain}

\begin{document}
\title{LoFIP Architecture}
 

\maketitle

\begin{abstract}
This is the second  task until July 23.
The goal of this document is to describe possible extensions and modification of the
``Leitstand'' architecture in LoFIP.
In general, we are interested in the architecture of a ``Leitstand'' (control tower)

\end{abstract}

%%%%%%%%%%%%%%%%%%%%%%%%%%%%%%%%%%%%%%%%
\section{The conceptual architecture of a control tower}
%%%%%%%%%%%%%%%%%%%%%%%%%%%%%%%%%%%%%%%%%%%%

According to our distinction of the different kinds of architectures and views on architectures,
we refer to the LoFIP architecture as conceptual (or as logical) architecture.

The LoFIP architecture consists of a \textbf{Front end}, a \textbf{back end} and 
the \textbf{control tower} (or control tower core part).

In the remainder of this document, we discuss the components of the \emph{control tower / ``Leitstand''}.

%%%%%%%%%%%%%%%%%%%%%%%%%%%%%%%%%%%%%%%%%%%%%%
\section{The MONITOR component}
%%%%%%%%%%%%%%%%%%%%%%%%%%%%%%%%%%%%%%%%%%%%%%%%%%%%

The monitor component is responsible for data harvesting, data filtering, data correlation, data aggregation
and prediction. 
The data harvesting / recognition obtains data directly from the IoT / IoS gateway. 
For the prediction, the monitor component uses observed data but also stored information from the
knowledge component.
A typical task of the monitor component is to keep track of the shipment of a certain container.
Predictions include issues like to provide estimations about the expected arrival of containers.

Collected and aggregated data is usually passed to the analyze component.

%%%%%%%%%%%%%%%%%%%%%%%%%%%%%%%%%%%%%%%%%%%%%%%%%%%%%%%%
\section{ANALYZE component}
%%%%%%%%%%%%%%%%%%%%%%%%%%%%%%%%%%%%%%%%%%%%%%%%%%%%%%%%

The analyze component is looking for divergences (deviations).
The goal is to observe situations and determine change needs if necessary.

 Divergences can be obtained by an observation and comparison
of component execution with respect to the behavior specifications (e.g., behavior specifications in terms of 
business process models}. A comparison determines the degree of deviation from a process specification.
The comparison might lead to two different results:
  \begin{enumerate}
		\item In case there are huge differences, these differences are forwarded to a process analyze and and
		simulation component. 
		 \item If the difference is not significant then it is just reported as a change for further improvement
		  (and investigations).
	\end{enumerate}

Consider the following example to get the difference:
If a trucker is late it is a deviation. If this might influence the overall process
then is is an exception  and there will be some priory assessment.

%%%%%%%%%%%%%%%%%%%%%%%%%%%%%%%%%%%%%%%%%%%%%%%%%%%%%%%%%
\section{PLAN component}
%%%%%%%%%%%%%%%%%%%%%%%%%%%%%%%%%%%%%%%%%%%%%%%%%%%%%%%%%

The plan component is recognizing any alternative executions and provides 
components to prioritize alternative executions. A changed plan represent / contains
a set of desired changes.


%%%%%%%%%%%%%%%%%%%%%%%%%%%%%%%%%%%%%%%%%%%%%%%%%%%%%%%%%
\section{EXECUTE component}
%%%%%%%%%%%%%%%%%%%%%%%%%%%%%%%%%%%%%%%%%%%%%%%%%%%%%%%%%

The execute component has to schedule and perform the necessary changes to the system. 
The execute component has to implement / realize the decisions made by the dispatcher.
This is done either based on data from the planing component (plans / panning steps) or
in terms of \emph{adaptations}.
Typical adaptations are to retract and modify existing transport orders and plans
like the route of the driver must be adapted once the dispatcher has changed the delivery planning.




%%%%%%%%%%%%%%%%%%%%%%%%%%%%%%%%%%%%%%%%%%%%%%%%%%%%%%%
\section{Basics: The MAPE-K Reference Model}
%%%%%%%%%%%%%%%%%%%%%%%%%%%%%%%%%%%%%%%%%%%%%%%%%

MAPE-K is a reference model for self-adaptive systems. 
IBM has suggested a reference model for autonomic control loops~\cite{IBM2005}.
The model is inspired by models
of autonomous agents. Based on (sensor) input, the component monitors the managed elements
and execute changes. The goal of actions (executions) is mainly specified by ECA-rules.

As already described above, the MAPE-K architecture provides essential building blocks
for self-adaptive systems. Each component provides a certain functionality:
(i)~the \emph{monitor} function (component) is responsible for collecting data,
then (ii)~the \emph{analyze} function takes into account the collected data and the current
system state, makes predictions of future states and presents alternative (prioritized) possible
future states,
afterwards (iii)~the \emph{planing} component constructs the needed (adaptation) actions that have been
proposed by the previous analyze function, and finally
(iv)~the \emph{execute} function is responsible for controlling the execution of a plan.

All functions share a common knowledge (base). 


%%%%%%%%%%%%%%%%%%%%%%%%%%%%%%%%%%%%%%%%%%%%%%%%%%%%%%%%%%%%%%
\section{Discussion: Control Tower and the MAPE-K Reference Model as Architecture}
%%%%%%%%%%%%%%%%%%%%%%%%%%%%%%%%%%%%%%%%%%%%%%%%%%%%%%%%%%%%%%%%%

At a first glance, the control tower architecture, which is based on the
MAPE-K reference model, is an appropriate architecture for adaptive systems.

As a next step, we want to generalize this architecture into two directions:
\begin{enumerate}
	\item Is there a more general description of the architecture that might even 
	be applicable to any `control tower' independent of a particular domain.
	 \item Until now, I cannot see the particular aspect of the ``Internet of Things''.
	 It is actually an architecture for a self-adaptive system instead of an Internet of Things application.
	
\end{enumerate}



%%%%%%%%%%%%%%%%%%%%%%%%%%%%%%%%%%%%%%%%%%%%%%%%%%%%%%%%%%%%%%%%%%%%%%%%%%

\bibliographystyle{plain}
\bibliography{bib-architecture}

\end{document}