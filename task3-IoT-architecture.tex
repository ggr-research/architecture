\documentclass{llncs} % , times

\usepackage{amssymb, amsmath, graphicx, ltxtable, longtable, tabularx, url, ragged2e, xspace, verbatim, fancybox,tikz}
\usepackage{scalefnt}
\usepackage{relsize}
\usepackage{hyperref}
%\usepackage{paralist}
%\usepackage{listings}

\newcommand{\ggr}[1]{\textcolor{magenta}{comment Gerd: \textit{#1}}}

\pagenumbering{arabic}
\pagestyle{plain}

\begin{document}
\title{Architecture for the Future Internet of Things (IoT)}
 

\maketitle

\begin{abstract}
This is the third  task until July 30.
The goal is to study basics and fundamental requirements of
an Future Internet architecture.
\end{abstract}


%%%%%%%%%%%%%%%%%%%%%%%%%%%%%%%%%%%%%%%%%%%%%%%%%%%%%%%%%%%%%
\section{Agenda -- Discussion July 23}

\begin{itemize}
	\item The distinction between these different architectures is fine.
   \item TODO: look into the different terms from teh RE book:
	       \begin{itemize}
					 \item \emph{Data}
					 \item \emph{functionality / functions}
					  \item \emph{behavior}
				 \end{itemize}
	\item The observation is right, that inside the core architecture (Leitstand) there 
	is nothing special wrt. the IoT / IoS.
	    \begin{itemize}
				\item this only holds for the Front end and Back end
				\item TODO: Look into the details within the architecture to find what is special
				         once we use / rely on IoT and IoS:
								(1)~ANALYZE and (2)~MONITOR component must be studied \\
								$\rightarrow$ this covers aspects like trust of information,
								 availability of services, distribution of service. \\
								$\rightarrow$ just think about everything that might
								   influence the ANALYSIS and MONITORING if data / information is obtained from
									  IoT and IoS
			\end{itemize}
  \item Further research questions:
	     How does MAPE-K goes along with user interfaces (UI) --> does this hold, is there something special???
	\item Further research questions:
	      Think about the motivation of MAPE-K in the used context in LoFIP:
				   (1)~business perspective (time value) and (ii) adaptive systems 
\end{itemize}

%%%%%%%%%%%%%%%%%%%%%%%%%%%%%%%%%%%%%%%%%%%%%%%%%%%%%%%%%%%%%%%%%%%%%
\section{Fundamentals and Terminology}

In requirements engineering, there are three kinds of requirements artefacts:
(1) goals, (2) scenarios and (3) \emph{solution oriented} requirements.
The last group of artefacts, i.e., the solution oriented requirements representation
should provide a \emph{conceptual} or \emph{logical} system solution
(e.g., in terms of data models and UML class diagrams).
Data models determine which data shall be represented in a system.
Finally, a \emph{behavior} model describe the state of a system and the externally visible behavior.

We summarize the different views for 
documenting \emph{solution-oriented requirements} according to~\cite{Pohl2010RE}.
There are three fundamental perspectives of software-intensive systems.
In requirements engineering, these aspects are also covered within the solution space.
We distinguish between the following perspectives.

\begin{itemize}
	\item The \textbf{data perspective} focuses on the data and information that is processes by
	 a system. The data structure, types and restrictions are of interest, but not
	 how this data is processed.
	
	
	  Data models are defined at the type (modeling) level. Thus, we talk about types and
		relationships between types. Data models can be represented by ER and EER diagrams
		  (with \textsf{Entities}, \textsf{Relationships} and \textsf{Attributes}). 
			(EER also support specialization and generalization of entity types.)
			 Class diagrams are also used to document data (e.g., a class refers to a set of 
		
    In order to build data model (e.g., an EER diagram), requirements engineers have to
		 identity \emph{entities}, \emph{relationships} and \emph{attributes}.
		
	\item The \textbf{functional perspective} defines how data is manipulated. This also includes input and output
	 relations between processing steps. The functional perspective is mainly documented by \emph{data flow} diagrams.
	Data flow diagrams can be refined, ending up with a hierarchy of different levels (of granularity).
	   
		
		
	\item The \textbf{Behavioral perspective} defines the (reactive) behavior of a system. In this perspective,
	  the external stimuli that the system receive and the reactions of the system as well as the relationships
		 between the input and the reactions are specified in this perspective. 
		This perspective is mainly expressed by automata and statecharts.
		Accordingly, the key parts of these models are \emph{state}, \emph{event} and \emph{state transition}.
		
		In order to build a statechart for a certain domain, the requirement engineer has to
		   define \emph{states} and \emph{transitions} between states.
		
\end{itemize}

To sum up, I doubt whether the \emph{domain-oriented view} refers to the \emph{data perspective} is
solution-oriented requirements engineering. Parts (entities) of the functional and behavioral perspective
also refer to the domain and there are relationships (mappings, alignments) between elements
of the different views (and these mappings express the relatedness (or even equality) of elements
in different perspectives.


%%%%%%%%%%%%%%%%%%%%%%%%%%%%%%%%%%%%%%%%%%%%%%%%%%%%%%%%%%%%%%%%%%%%%%%%%%

\bibliographystyle{plain}
\bibliography{bib-architecture}

\end{document}