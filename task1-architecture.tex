\documentclass{llncs} % , times

\usepackage{amssymb, amsmath, graphicx, ltxtable, longtable, tabularx, url, ragged2e, xspace, verbatim, fancybox,tikz}
\usepackage{scalefnt}
\usepackage{relsize}
\usepackage{hyperref}
%\usepackage{paralist}
%\usepackage{listings}

\newcommand{\ggr}[1]{\textcolor{magenta}{comment Gerd: \textit{#1}}}

\pagenumbering{arabic}
\pagestyle{plain}

\begin{document}
\title{Basic Definitions for Software Architecture}
 

\maketitle

\begin{abstract}
This is the first task until next week, July 16.
The goal of this document is to describe and clarify basic terms of software architectures.

\end{abstract}

%%%%%%%%%%%%%%%%%%%%%%%%%%%%%%%%%%%%%%%%
\section{Fundamentals on Software Architecture}
%%%%%%%%%%%%%%%%%%%%%%%%%%%%%%%%%%%%%%%%%%%%


A software architecture divides an application domain into well-defined and separated pieces of
responsibility.

\begin{definition}[Architecture]
An architecture is defined as the fundamental organization of a system,
embodied in its components, their relationships to each other and
the environment and the principles governing the design and
evolution (IEEE standard).
\end{definition}

An alternative definition of software architecture is, according to~\cite{Buschmann1998POS}:
``the structure of structures of the system, which comprise
software components, the externally visible properties of those components and the relationships
among them.''

An architecture of a system is supposed to give answers to several questions
regarding the \emph{functionality} of the system's elements,
the \emph{interaction} between elements,
the \emph{operational features} of the elements and
the \emph{deployment} and \emph{hardware} (cf.~\cite{Rozanski2005SSA}).

Due to the complexity of a system, a monolithic architecture might
be incomplete, incorrect or hard to manage.
A common principle is to partition an architecture into separate
but interrelated \emph{views}.
Each view describes a different aspect of the system.  
Bass et al.~\cite{Bass1998SAI} use the term ``architecture structure'' as synonym to ``view''.

\begin{definition}[View]
A view is a representation of a whole system from the perspective of a related set of concerns.
\end{definition}




Kruchten~\cite{Kruchten1995T41} introduced the 4+1 view model on architectures,
depicted in Fig.~\ref{fig:viewmodel}.
The model consists of four \emph{architectural views}:
\begin{itemize}
	\item \textbf{Logical view}, also called \textbf{functional view}~\cite{Clements2001ESA}:
	        This view is an abstraction of the system functions and their relationships. The audience
					 of this view are \emph{domain} engineers and end users.
	\item \textbf{Process view}, also called \textbf{concurrency view}
	                or \textbf{thread view}~\cite{Clements2001ESA}:
									 This view describes system processes and threads and how they communicate with each other.
									This view is for system integrators, tester and engineers who are
									responsible for the performance and availability of the system.
	 \item \textbf{Deployment view}, also called \textbf{implementation view}~\cite{Clements2001ESA}:
                     
	 \item \textbf{Physical view}
\end{itemize}

The fifth part of the view model is the use case view. It is a central part of the model since
all architectural decisions must be based on use cases.

\begin{figure}%
\begin{center}
\includegraphics[width=\columnwidth]{figs/viewmodel.png}%
\caption{4+1 view model}%
\label{fig:viewmodel}%
\end{center}
\end{figure}

This basic idea of the view model is generalized in the IEEE standard  by the notion of \emph{viewpoints}.
According to~\cite{Rozanski2005SSA}, a viewpoint is defined as follows.

\begin{definition}[Viewpoint]
A viewpoint is a collection of patterns and conventions to construct one type of view.
It describes the stakeholders whose concerns  are reflected in the viewpoint
and the guidelines, principles and templates for building its view.
\end{definition}

The term \emph{perspective} or \emph{architectural perspective} is used in~\cite{Rozanski2005SSA}
to refer to \emph{non-functional requirements}. This perspective is orthogonal to
the views of an architecture. Perspectives contain,
among other aspects, security, performance, availability and usability.


Different terms for types of software architectures and views are used in a study 
on industrial applications, mainly in the realm of real-time and embedded systems.
In~\cite{DBLP:conf/icse/SoniNH95}, Soni et al.~distinguish between four types of architectures.
This distinction is based on a survey of (industrial) systems.
 \begin{itemize}
	 \item The \textbf{conceptual architecture} describes a system by design elements
	and relationships among them. The architecture is independent of the concrete implementation.
		\item The \textbf{module architecture} is focused on the decompositions of the
		  system into sub-systems (and of elements into sub-elements). 
			 This architecture already covers implementation decisions,
			 but independent of a particular programming language.
			 
	  \item The \textbf{execution architecture} describes the dynamic structure of the system
		        by run-time elements like \emph{tasks}, \emph{processes}, etc.
						The execution architecture is used for performance and scalability analysis.
		 \item The \textbf{code architecture} describes the organization of source code in
		       libraries, directories and files.
					This architecture is usually not considered as a software architecture.
 \end{itemize}

This distinction is inspired by the phases of a development process:
\emph{design time}, \emph{implementation time}, \emph{build time} and \emph{run time}.

Regarding to the first classification, the \emph{conceptual architecture} and \emph{module architecture}
refer to the \emph{logical view} (especially the module architecture).



The overall architecture model (big picture) could be summarized as follows:
\ggr{some of these terms and their organization is based on
      presentations (lecture slides (TUM) and conference tutorials)  --- I have not found 
			 literature yet.}
			
\begin{enumerate}
  \item \textbf{Conceptual architecture:}
				\begin{itemize}
           	\item \textbf{Usage Level, User functionality, customer functions:}
						   This is an organized    description of intended functionality:
							 feature hierarchies,  feature interactions
            \item \textbf{Logical architecture (functional architecture)}: 							
						      Hierarchical decomposition of the system into logical components
  				\end{itemize}
	\item \textbf{Technical architecture:}
           \begin{itemize}
               	\item \textbf{Software architecture:}
								      \begin{itemize}
												\item Design time software architecture
												 \item run-time software architecture
											\end{itemize}
								\item Hardware Architecture
								\item Deployment
            \end{itemize}
\end{enumerate}

Fig.~\ref{fig:comprehensive} illustrates this categorization.

\begin{figure}%
\begin{center}
\includegraphics[width=\columnwidth]{figs/comprehensive-architecture.png}%
\caption{Comprehensive architecture}%
\label{fig:comprehensive}%
\end{center}
\end{figure}


% In the realm of architectures for embedded systems, there is some work
% on the realization / implementation of domain aspects  in
% a technical architecture.


% WICHTIG
 % customer functional structeure -->  logical architecture --> software architecture   --> deployment  (hardware architecture)


% \begin{itemize}
% 	\item The \textbf{conceptual architecture (view)} describes the basic components and relationships,
% 	 but without further specifications like interfaces.
%	\item The \textbf{logical architecture} is a more fine-grained specification
%	\item  The \textbf{distribution architecture} refers to the \emph{physical} view in the
%	              4+1 view-model.
% \end{itemize}


%%%%%%%%%%%%%%%%%%%%%%%%%%%%%%% maybe remove this 
% The goal of views is to reduce the complexity and to focus on specific aspects.
% According to Broy, we can distinguish between kinds of vies:
%   \begin{itemize}
%	 	\item Specification Views: they are used to analyze and design a system
%		       \begin{itemize}
%			 			\item Fachliche Sicht (domain view)
%				 		\item Technische Sicht (technical view)
%					 	\item Verteilungssicht (distribution view)
%					 \end{itemize}
%		  \item Development views: they are used for the engineering and development
%	 \end{itemize}
%	
%	 \vspace{1em}
%	 \textbf{Fachliche Sicht}: Describe the concepts of an application domain independent of
%	        any software, implementation or platform.
%		 		  It contains components of the domain including interfaces and behavior descriptions.
%					 
% \vspace{1em}
% \textbf{Technical view:} describes a software, system, platform or programming language specific
% solution. It describes technical components


	
	
	%%%%%%%%%%%%%%% Broy technical report TUM-I0816
%	There are several levels of abstractions of the essential elements of an architecture model:
%	\begin{itemize}
%		\item Usage level (Nutzungsebene): describes the usage and functions and relations between functions.
%		\item Logical Architecture: Components, subcomponents and their relationships and interaction relations
%		\item Technical architecture: Hardware and software units and their embedding
%	\end{itemize}
	
	
	

\ggr{the terms architecture and view are used interchangeably in the remainder of this document.}



%%%%%%%%%%%%%%%%%%%%%%%%%%%%%%%%%%%%%%%%%%%%%%%%%%%%%%%%%%%%
\section{Conceptual Architecture}
%%%%%%%%%%%%%%%%%%%%%%%%%%%%%%%%%%%%%%%%%%%%%%%%%%%%%%%%%%%%%%

The conceptual architecture is an initial architecture within the development process.
It serves as the first response to stakeholder's needs.

The conceptual architecture (often called conceptual view) is focused
on the functionality and on quality attributes of a system. The focus is on the domain-level,
independent of a concrete implementation.
It describes functionality, functions and sub-functions from a stakeholder and domain-oriented
point of view.

Tekinerdogan et al.~\cite{Tekinerdogan2008SAR} use the term ``conceptual architecture'',
without any definition and explanation of this term. The modules in the model are specific for the domain,
in this case for digital TV.
%\ggr{This directly leads to the question }


The \emph{conceptual architecture} of a system is described in~\cite{Reekie2006ASA} as follows.
The conceptual architecture is focused on understanding a system in terms of
domain-level responsibilities. It should cover how the system's architecture meets
the immediate needs of its stakeholders.
A component covers domain responsibilities. 
The architecture originally describes the functional specification
and is then iteratively further developed to cover additional aspects like performance, security,
criticallity and so on.
The conceptual architecture is influenced by non-functional concerns,
but it is too abstract of a comprehensive treatment of quality aspects.

\paragraph{Building blocks (elements).}

The conceptual architecture consists of the following elements:
\begin{itemize}
	\item \textbf{Conceptual components} represent the key functional elements and their relationships.
	They are described in terms of \emph{domain-level responsibilities}, i.e., they refer to
	concepts of the problem domain, rather than on the implementation of the components.
	A \emph{conceptual component} can be considered as a set of related \emph{responsibilities}.
	Quite often, conceptual components do not have a correspondent counterpart in the software.
	\item \textbf{Connectors} between components indicate an exchange of information.
	\item \textbf{External Interfaces} enable the connection to external systems. 
\end{itemize}

\paragraph{Design of the Conceptual Architecture.}
The design of an architecture is an iterative process. The conceptual architecture is based on the functional requirements of the system.
Thus, the conceptual architecture helps  at the beginning of the development process to
understand the system. In essence, we can outline the following steps:
 \begin{enumerate}
	 \item Create an initial conceptual architecture based on a description of the problem domain.
	         This is similar to an ontology in which concepts (abstract concepts) and their relationships
					      are described.
	 \item Elaborate the architecture based on the functionality (functional requirements).
	       For instance, properties of components are identified, the data structure is identified.
				 By doing this, we add semantics to the components.
	 \item Refine the architecture with respect to quality attributes (non-functional requirements).
	 \item Iterate the second and third step.
 \end{enumerate}


%%%%%%%%%%%%%%%%%%%%%%%%%%%%%%%%%%%%%%%%%%
\section{Logical Architecture / Logical View}
%%%%%%%%%%%%%%%%%%%%%%%%%%%%%%%%%%%%%%%%%%%%%%%%%%%%%%%%%%%%%%%%%%%

The conceptual architecture (or conceptual view) does not provide (implementation-oriented) details
like interfaces of components.
In contrast, the logical architecture (or logical view) gives more details about the components
and their relationships, including interfaces and communication specifications.
In essence, it describes the functionality of the components.

%\subsection{Work from Broy et al.}
% Broy
The \textbf{logical architecture}  describes \emph{logical components}.
A logical component is a unit that is responsible for some functionality.
%In general, there are n-m relationships between services of the usage level and logical components.
%% WICHTIG -- it seems that USAGE LEVEL == CONCEPTUAL VIEW
A logical component has a syntactic and semantic interface.
The logical architecture contains structural information
and the division of the system into communicating components.
Furthermore, the behavior of logical components is \emph{total},
i.e., for each input there must be some reaction of the system.
Thus, the behavior can be simulated.

Due to the total definition of behavior, the logical architecture is a model that describes
the complete functionality of the system. It it the first model (in the level hierarchy) 
that describes the whole functionality.  

Constituents of the logical architecture are the following:
(i) Logical components are units (of functions).
	Formally, a logical components consists of a syntactic and semantic interface. The syntactic interface
	describes types of the port (logical types). The semantic interface specifies the behavior, i.e., for
	each syntactic valid input there is a well-defined output described. The behavior is state-oriented.
(ii) Ports connect components to their environment.
(iii) Channels describe direct communication between components. A channel describes an abstract
communication by connecting two ports.
	


%%% logisches Systemmodell
%  \ggr{remove the following definition}
%  \begin{definition}[Logical System Model]
%  Die logische Systemhierarchie beschreibt ein system als eine Hierarchie von Teilsystemen.
%  Dabei hat jedes System und Teilsystem ein Schnittstellenverhalten.
% \end{definition}

A logical architecture is a user view. In the development process, a logical architecture
must be transformed to a software architecture.

%%%%%%%%%%%%%%%%%%%%%%%%%%%%%%%%%%%%%%%%%%%%%%%%%%%%%%%%%%%
\section{Domain-oriented Architecture (Fachliche Architektur)}
%%%%%%%%%%%%%%%%%%%%%%%%%%%%%%%%%%%%%%%%%%

The domain-oriented architecture / business architecture (or `fachline Architectur')
is \emph{orthogonal} to the previously introduced architectures and views.
Most related is the area of enterprise architecture frameworks like TOGAF~\cite{togaf91}
and Zachman~\cite{Zachman1987AFF}.

The domain-oriented architecture describes the concepts of an application domain
independent of any software, implementation or platform.
%It contains components of the domain including interfaces and behavior descriptions.

\ggr{check this:} From my understanding, a domain-oriented architecture
can be a conceptual and logical architecture (view),
but it must cover the entities of the corresponding application domain.
A domain-oriented architecture seems to be closer related to the conceptual architecture
as to the logical architecture since the key focus is on
%%%%%%%%%%%%%%%%%%%%%%%%%%%%%%%%%%%%%%%%%%%%%%%%%%%%%%%%%%%%
\section{Technical Architecture}
%%%%%%%%%%%%%%%%%%%%%%%%%%%%%%%%%%%%%%%%%%%%%%%%%%%%%%%%%%%%%%


This is the level with the lowest degree of abstraction. 
It describes a software, system, platform or programming language specific solution.
There is still some level of abstraction in order to analyze real-time achievements,
simulate behavior.

%%%%%%%%%%%%%%%%%%%%%%%%%%%%%%%%%%%%%%%%%%%%%%%%%%%
\section{The Architecture in LoFIP}
\label{sec:lofip}
%%%%%%%%%%%%%%%%%%%%%%%%%%%%%%%%%%%%%%%

The ``Leitstand'' architecture in LoFIP is a logical architecture.

As previously mentioned, the domain-oriented architecture  is orthogonal to the different
architecture views.
An architecture is a domain-oriented architecture (fachliche Architektur)
if its elements refer to concrete entities of the domain and
it must cover all entities that are part of the system's requirements and use cases,
i.e., it must be complete~\cite{Posch2004BS}.


The ``Leitstand'' architecture contains domain entities like ``Transportpl{\"a}ne'',
``LKW Fahrer'' and ``Disponent'', but it is not complete. 


%%%%%%%%%%%%%%%%%%%%%%%%%%%%%%%%%%%%%%%%
%\section{More aspects about the term ``domain''}
%%%%%%%%%%%%%%%%%%%%%%%%%%%%%%%%%%%%%%%%%%%%%%%%%%%%

%  when talking about the domain, domain models are key building blocks
%    when talking about domain modeling -- FODA is an essential means to build domain models,
%           i.e., to describe a domain of interest.
%                 a domain model is the products of a domain analysis



\bibliographystyle{plain}
\bibliography{bib-architecture}

\end{document}